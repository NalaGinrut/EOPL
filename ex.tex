\documentclass[11pt,a4paper]{report}
\usepackage{amsfonts}
\usepackage{amsthm}
\usepackage{color}
\usepackage{listings} 

\begin{document}
\title{Execise of EOPL}
\author{Mu Lei known as NalaGinrut\\nalaginrut@gmail.com}
\lstdefinestyle{custom-lisp}{
  belowcaptionskip=1\baselineskip,
  breaklines=true,
  frame=L,
  xleftmargin=\parindent,
  language=Lisp,
  showstringspaces=false,
}
\maketitle
\newtheorem{defn}{Definition}[section]
\chapter{Execise 1.1}
\section{$\{3n + 2 \mid n \in \mathbb{N} \}$}
\begin{defn}[\bf top-down]
A natural number $n$ is in $S$ if and only if
\begin{enumerate}
\item 
$n=2$, or
\item 
$\frac{n-2}{3} \in S.$
\end{enumerate}
\lstset{language=Lisp,escapechar=@,style=custom-lisp}
\begin{lstlisting}[frame=single]  % Start your code-block

(define (func x)
  (if (= x 2)
      2 
      (if (natural? (/ (- x 2) 3))
           (func (/ (- x 2) 3)) 
           #f)))
\end{lstlisting}
{\bf hint:} Find the first elem, check each elem iterately, when the final result of {\color{red} inverse function} is the first elem, accept it.
The first elem equal to recursive-exit.
\end{defn}
\begin{defn}[\bf bottom-up]
Define the set $S$ to be the smallest contained in $\mathbb{N}$ and satisfying the following two properties:
\begin{enumerate}
\item
$2 \in S$, and
\item
if $n \in S$, then $ 3n+2 \in S. $
\end{enumerate}
\lstset{language=Lisp,escapechar=@,style=custom-lisp}
\begin{lstlisting}[frame=single]  % Start your code-block

(define (func n) 
  (let lp((i 0) (ret '()))
    (cond 
      ((>= i n) (reverse ret))
      (else (lp (+ 2 (* 3 i)) 
                (cons (+ (* 3 i) 2) ret))))))
\end{lstlisting}
{\bf hint:} Find the first minimum set. and add each check each elem from the minimum set which satisfying the inverse function.
\end{defn}

\begin{defn}[\bf rules of inference]
to be continued...
\end{defn}

\section{problem 2}
\begin{defn}
asdf
\end{defn}
$ a \not\in S $

\end{document}
\documentstyle{article}